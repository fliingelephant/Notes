% !TEX program = xelatex

\documentclass[UTF8]{ctexart}
\usepackage{amsmath, amssymb, amsthm}
\usepackage{bm}
\usepackage{booktabs}
\usepackage{geometry}
\usepackage{hyperref}

\newcommand{\btheta}{\boldsymbol{\theta}}
\geometry{a4paper, margin=2cm}
\title{量子几何张量 vs Fisher 信息矩阵\\[0.5em]\large 从 KL 散度到 Fubini-Study 距离,从自然梯度到随机重构的自洽推导}
\author{}
\date{}
\begin{document}
\maketitle

\begin{abstract}
经典信息几何中,KL 散度在小参数扰动下的二阶展开诱导出 Fisher-Rao 度量,其度量张量即 Fisher 信息矩阵(FIM);对应的"最速下降"给出自然梯度法。量子纯态的物理态是射线(ray),存在全局相位与整体归一化的规范冗余,因而不能直接把"对输出的欧氏内积/协方差"当作度量;正确的距离应定义在射影 Hilbert 空间上,即 Fubini-Study(FS)距离。FS 距离的二阶展开给出量子几何张量(QGT):其实部是 FS 黎曼度量,虚部是 Berry 曲率。进一步地,在以 FS 范数约束的最速下降问题中,参数更新满足线性方程 $S\delta\theta = -\eta g$,这正是变分蒙特卡洛与神经网络量子态训练中常用的随机重构(SR)/量子自然梯度(QNG)更新。本文从经典部分开始,给出 KL 一阶项为零、二阶项等于 score 协方差的完整证明,并在清晰区分"概率输出"和"复振幅输出"的基础上,逐步引入 FS 距离、QGT 的规范不变定义及其与 SR 的几何推导。文末用两张对照表总结经典与量子概念的对应关系,以及常见优化算法与其隐含的度量/流形解释(包括 Adam/RMSProp 等对角近似)。
\end{abstract}

\tableofcontents

\section{经典信息几何:KL 散度的局部展开与 Fisher-Rao 度量}

\subsection{从KL 散度到Fisher信息矩阵}
设 $p_{\btheta}(x)$ 是由一组参数 $\btheta = (\theta^{1}, \dots, \theta^{d})$ 参数化的概率分布(离散与连续情形统一记作 $\sum_x$)。两分布之间的差异由 Kullback-Leibler (KL) 散度刻画:
\begin{equation}
D_{\mathrm{KL}} \left(p_{\btheta} \| p_{\btheta^\prime}\right) := \sum_x \, p_{\btheta}(x) \ln \frac{p_{\btheta}(x)}{p_{\btheta^\prime}(x)}
=\sum_x \, p_{\btheta}(x) \ln p_{\btheta}(x) - \sum_x \, p_{\btheta}(x) \ln p_{\btheta^\prime}(x).
\end{equation}
注意KL散度不对称,所以严格来说它不是一种“距离”。经典信息几何把分布族 $\{p_{\btheta}\}$ 看作流形上的点集,我们仍希望在该流形上定义“距离”。一个自然思路是:当两分布非常接近时,用某种距离度量来近似 KL 散度。

考虑参数的微小变动 $\btheta^\prime = \btheta + \delta \btheta$,对KL散度做泰勒展开(即展开 $\ln p_{\btheta+\delta \btheta}(x)$):
\begin{align*}
D_{\mathrm{KL}} \left(p_{\btheta} \| p_{\btheta + \delta \btheta}\right)&=\sum_x \, p_{\btheta}(x) \ln p_{\btheta}(x) - \sum_x \, p_{\btheta}(x) \ln p_{\btheta+\delta \btheta}(x)\\
&=\sum_x \, p_{\btheta}(x) \underbrace{\left[\ln p_{\btheta}(x) - \ln p_{\btheta}(x)\right]}_{=0} &\text{(零阶项)}\\
&- \delta \theta^i \underbrace{\sum_x p_{\btheta}(x) \partial_i \ln p_{\btheta}(x)}_{=0} &\text{(一阶项)}\\
&- \frac{1}{2} \delta \theta^i \delta \theta^j \underbrace{\sum_x p_{\btheta}(x) \partial_i \partial_j \ln p_{\btheta}(x)}_{:=-I_{ij}(\btheta)} &\text{(二阶项)}\\
&+ \mathrm{O}(\|\delta \btheta\|^3)
\end{align*}
其中
\begin{itemize}
    \item 零阶项为零(相同分布的KL散度为零)
    \item 一阶项为零:将概率归一化条件$\sum_x p_{\btheta}(x) = 1$ 对 $\theta^j$ 求导给出
\begin{equation}
    \partial_j \sum_x p_{\btheta}(x) \overset{\text{求导积分换序}}{=} \sum_x \partial_j p_{\btheta}(x) = \sum_x p_{\btheta}(x) \, \partial_j \ln p_{\btheta}(x) = 0.
    \tag{7}
\end{equation}
即一阶项也为零。这里隐含了一个条件:对于 $p_{\btheta} (x)$,求导和积分可以换序,也即\textbf{正则化条件}。

    \item 二阶项:对式 (7) 再关于 $\theta^i$ 求导,
    \begin{align*}
    0 &= \partial_i \sum_x p_{\btheta}(x) \, \partial_j \ln p_{\btheta}(x)\\
    &= \sum_x \partial_i \bigl[p_{\btheta}(x) \, \partial_j \ln p_{\btheta}(x)\bigr]\\
    &= \sum_x \bigl[ p_{\btheta}(x) \, \partial_i \ln p_{\btheta}(x) \cdot \partial_j \ln p_{\btheta}(x) + p_{\btheta}(x) \, \partial_i \partial_j \ln p_{\btheta}(x) \bigr].
    \end{align*}
    这里隐含的条件是对于 $\partial_j \ln p_{\btheta} (x)$,求导和积分可以换序。
    移项得
    \begin{equation}
    \sum_x p_{\btheta}(x) \, \partial_i \ln p_{\btheta}(x) \cdot \partial_j \ln p_{\btheta}(x) = -\sum_x p_{\btheta}(x) \, \partial_i \partial_j \ln p_{\btheta}(x).
    \tag{9}
    \end{equation}
    将式子左边定义为 \textbf{Fisher 信息矩阵}
    \begin{equation}
    I_{ij}(\btheta) := \sum_x p_{\btheta}(x) \, \partial_i \ln p_{\btheta}(x) \cdot \partial_j \ln p_{\btheta}(x).
    \tag{10}
    \end{equation}
    观察到式子右边是$\ln p_{\btheta}(x)$对应的Hessian矩阵的负期望值,即
    \begin{equation}
        \text{Fisher 信息矩阵} = - \mathbb{E}_{p_{\btheta}}[ \text{Hessian of } \ln p_{\btheta}(x) ]
    \end{equation}
\end{itemize}

代入前述KL散度的泰勒展开,得到
\begin{equation}
D_{\mathrm{KL}}(p_{\btheta} \| p_{\btheta+\delta\btheta}) \approx \frac{1}{2} \delta\theta^i \delta\theta^j I_{ij}(\btheta) = \frac{1}{2} \delta\btheta^\top I(\btheta) \, \delta\btheta.
\tag{11}
\end{equation}
这表明在局部我们可以用 $I_{ij}$ 作为距离来近似KL散度。这是KL散度诱导的一个黎曼度规,称为\textbf{Fisher-Rao 度规},从而我们可以将分布们看作一个黎曼流形,并谈论其上的距离。

\begin{table}[ht]\centering
\caption{KL散度泰勒展开各阶项}
\begin{tabular}{c c p{6cm} c}
\toprule
\textbf{阶数} & \textbf{值} & \textbf{推导} & \textbf{条件} \\
\midrule
0 & $0$ & $D_{\mathrm{KL}}(p \| p) = 0$ & — \\
1 & $0$ & $\partial_i \bigl(\sum_x p_{\btheta}(x) - 1\bigr) = 0$ & 积分、求导可换序 \\
2 & $\frac{1}{2} \delta\btheta^\top I(\btheta) \, \delta\btheta$ & $\partial_i \partial_j \bigl(\sum_x p_{\btheta}(x) - 1\bigr) = 0$ & 积分、求导可换序 \\
\bottomrule
\end{tabular}
\end{table}

\subsection{Score 向量}
为了表述方便,我们定义 \textbf{score 向量}(对数似然梯度)
\[
s_i(x) := \partial_i \ln p_{\btheta}(x),
\]
并引入期望记号 $\mathbb{E}_{p_{\btheta}}[f(x)] := \sum_x p_{\btheta}(x) \, f(x)$。则前述各式可简写为:
\begin{itemize}
    \item 式 (7):$\mathbb{E}[s_i] = 0$(score 均值为零);
    \item 式 (10):$I_{ij} = \mathbb{E}[s_i \, s_j]$(Fisher 信息 = score 外积的期望)。
\end{itemize}
由于 $\mathbb{E}[s_i] = 0$,$I_{ij}$ 同时也是 score 的协方差矩阵:$I_{ij} = \mathrm{Cov}(s_i, s_j)$。

\textbf{实践提醒:}上述期望取自模型分布 $p_{\btheta}$。在机器学习中,损失函数的 Hessian $H_{ij} = \partial_i \partial_j \mathcal{L}$ 一般不等于 Fisher 信息矩阵。二者相等当且仅当:(1) 损失为负对数似然 $\mathcal{L} = -\ln p_{\btheta}(x)$;(2) 期望取自模型分布而非数据分布。

\subsection{流形上的最速下降}
考虑优化损失函数 $\mathcal{L}(\btheta)$,其梯度为 $g = \nabla_{\btheta} \mathcal{L}$。普通梯度下降以欧氏范数 $\|\delta\btheta\|_2$ 约束步长,但这不具备重参数化不变性。若改用 Fisher-Rao 度规 $\|\delta\btheta\|_I^2 = \delta\btheta^\top I(\btheta) \, \delta\btheta$ 约束步长,则最速下降问题为
\[
\max_{\delta\btheta} \; -g^\top \delta\btheta \quad \text{s.t.} \quad \delta\btheta^\top I(\btheta) \, \delta\btheta = \epsilon^2.
\]
用 Lagrange 乘子法,得
\begin{equation}
I(\btheta) \, \delta\btheta = -\eta g, \quad \Rightarrow \quad \delta\btheta = -\eta I(\btheta)^{-1} g.
\tag{12}
\end{equation}
这就是 Amari 提出的自然梯度(Natural Gradient Descent, NGD)更新公式。

\section{从经典概率分布到量子态}

\subsection{经典模型 vs. 量子模型:输出与冗余}
经典概率模型直接给出实非负、归一化的概率分布 $p_{\theta}(x)$。
量子模型(例如参数化变分 Ansatz)返回态向量 $|\Psi(\theta)\rangle$,概率通过 Born 规则
\[
P(x)=\frac{|\langle x|\Psi(\theta)\rangle|^{2}}{\langle\Psi(\theta)|\Psi(\theta)\rangle}.
\]
乘以任意非零复数 $c$ 不会改变上式,因此 $|\Psi\rangle\sim c|\Psi\rangle$(全局相位与整体幅值)构成 \textbf{规范冗余}。

\subsection{Fisher 信息为何不能直接照搬到复振幅}
若把复振幅当作“输出”并生搬 Fisher 信息,会遇到:
\begin{enumerate}
    \item 振幅不可直接观测,无法在概率空间中定义期望;
    \item 全局相位方向不影响概率,却令信息矩阵沿该方向奇异;
    \item 整体归一化同理,也会被误计入。
\end{enumerate}
因此须先除去规范自由度,再谈度量。

\subsection{正确路线:Fubini--Study 距离 $\Rightarrow$ 量子几何张量}
纯态的物理空间是射影 Hilbert 空间 $\mathbb{CP}^{n-1}$。其天然距离是 \emph{Fubini--Study (FS) 距离}。
对 FS 距离在参数空间做二阶展开产生 \textbf{量子几何张量} (QGT):实部给出 FS 黎曼度量,虚部给出 Berry 曲率——与经典 “KL $\Rightarrow$ Fisher” 完全对应。

\section{Fubini-Study 距离与量子几何张量(QGT):规范不变定义与推导}

\subsection{Fubini-Study 距离:射影 Hilbert 空间上的天然距离}
对于两个(不必归一化的)态 $|\Psi\rangle$ 与 $|\Phi\rangle$,Fubini-Study 距离定义为
\begin{equation}
d_{\mathrm{FS}}(|\Psi\rangle, |\Phi\rangle) = \arccos \frac{|\langle \Psi | \Phi \rangle|}{\sqrt{\langle \Psi | \Psi \rangle \langle \Phi | \Phi \rangle}}.
\tag{13}
\end{equation}
显然 $d_{\mathrm{FS}}$ 对 $|\Psi\rangle \to c|\Psi\rangle$ 和 $|\Phi\rangle \to c'|\Phi\rangle$ 不变,因此是定义在射影空间上的距离。

\subsection{从 FS 距离到 QGT:二阶展开的详细计算}
设 $|\Psi(\theta)\rangle$ 是参数化的态,考虑 $|\Psi(\theta)\rangle$ 与 $|\Psi(\theta+\delta\theta)\rangle$ 的 FS 距离平方。记
\[
F(\delta\theta) = \frac{|\langle \Psi(\theta) | \Psi(\theta+\delta\theta) \rangle|^{2}}{\langle \Psi(\theta) | \Psi(\theta) \rangle \langle \Psi(\theta+\delta\theta) | \Psi(\theta+\delta\theta) \rangle}.
\]
在 $\delta\theta = 0$ 处 $F(0) = 1$,展开 $F$ 到二阶:
\[
F(\delta\theta) \approx 1 - \delta\theta^{\mu} \delta\theta^{\nu} \Re(Q_{\mu\nu}) + O(\|\delta\theta\|^{3}),
\]
其中 $Q_{\mu\nu}$ 即为量子几何张量。由 $d_{\mathrm{FS}}^{2} \approx 2(1 - \sqrt{F})$ 可得
\begin{equation}
\mathrm{d}s_{\mathrm{FS}}^{2} = g_{\mu\nu} \, \mathrm{d}\theta^{\mu} \mathrm{d}\theta^{\nu}, \quad g_{\mu\nu} = \Re(Q_{\mu\nu}).
\tag{14}
\end{equation}

\subsection{QGT 的规范不变定义与分解:度量 + Berry 曲率}
对于依赖参数 $\theta$ 的(不必归一化的)纯态 $|\Psi(\theta)\rangle$,QGT 定义为
\begin{equation}
Q_{\mu\nu} = \frac{\langle \partial_{\mu} \Psi | (I - \Pi_{\Psi}) | \partial_{\nu} \Psi \rangle}{\langle \Psi | \Psi \rangle}, \quad \Pi_{\Psi} = \frac{|\Psi\rangle \langle \Psi|}{\langle \Psi | \Psi \rangle}.
\tag{15}
\end{equation}
其中 $|\partial_{\mu}\Psi\rangle \equiv \partial_{\theta^{\mu}} |\Psi(\theta)\rangle$。投影算符 $I - \Pi_{\Psi}$ 扣除了导数态在 $|\Psi\rangle$ 方向的分量,保证规范不变性。

展开式 (15) 可得
\begin{equation}
Q_{\mu\nu} = \frac{\langle \partial_{\mu}\Psi | \partial_{\nu}\Psi \rangle}{\langle \Psi | \Psi \rangle} - \frac{\langle \partial_{\mu}\Psi | \Psi \rangle \langle \Psi | \partial_{\nu}\Psi \rangle}{\langle \Psi | \Psi \rangle^{2}}.
\tag{16}
\end{equation}
$Q_{\mu\nu}$ 是复值张量,可分解为
\begin{equation}
Q_{\mu\nu} = g_{\mu\nu} + \frac{i}{2} F_{\mu\nu}, \quad g_{\mu\nu} = \Re(Q_{\mu\nu}), \quad F_{\mu\nu} = 2\Im(Q_{\mu\nu}).
\tag{17}
\end{equation}
其中对称的实部 $g_{\mu\nu}$ 是 Fubini-Study 度量在参数空间的表示,反对称的虚部 $F_{\mu\nu}$ 是 Berry 曲率。

\subsection{测量基表示与 Monte Carlo 估计:QGT 作为"对数导数的协方差"}
在计算基 $\{|x\rangle\}$ 下,定义 $\Psi(x) = \langle x | \Psi \rangle$ 及对数导数
\begin{equation}
O_{\mu}(x) = \frac{\partial_{\mu} \Psi(x)}{\Psi(x)} = \partial_{\mu} \ln \Psi(x).
\tag{18}
\end{equation}
则 QGT 可写为
\begin{equation}
Q_{\mu\nu} = \mathbb{E}_{P}[O_{\mu}^{*} O_{\nu}] - \mathbb{E}_{P}[O_{\mu}^{*}] \mathbb{E}_{P}[O_{\nu}],
\tag{19}
\end{equation}
其中期望取自 Born 概率 $P(x) = |\Psi(x)|^{2} / \langle \Psi | \Psi \rangle$。这一形式与经典 Fisher 信息的"score 协方差"形式完全类比,但注意这里的 $O_{\mu}$ 是复值。

\section{随机重构 SR 与最速下降:FS 范数约束下的严格推导,并与经典自然梯度对照}

\subsection{SR 的基本线性方程}
随机重构(Stochastic Reconfiguration, SR)是变分蒙特卡洛中常用的参数更新方法,其核心方程为
\begin{equation}
S(\btheta) \, \delta\btheta = -\eta g,
\tag{20}
\end{equation}
其中 $S(\btheta)$ 是 QGT 的实部(或其对称化/正则化版本),$g = \nabla_{\btheta} E$ 是能量梯度。

\subsection{证明:FS 范数约束的最速下降推出 SR}
考虑优化能量 $E(\btheta) = \langle \Psi(\btheta) | \hat{H} | \Psi(\btheta) \rangle / \langle \Psi | \Psi \rangle$,其梯度为 $g = \nabla_{\btheta} E$。以 FS 度量约束步长范数:
\[
\max_{\delta\btheta} \; -g^\top \delta\btheta \quad \text{s.t.} \quad \delta\btheta^\top S(\btheta) \, \delta\btheta = \epsilon^2.
\]
用 Lagrange 乘子法,令
\[
\mathcal{L}(\delta\btheta, \lambda) = -g^\top \delta\btheta + \frac{\lambda}{2}(\delta\btheta^\top S \, \delta\btheta - \epsilon^2).
\]
对 $\delta\btheta$ 求导并令其为零:
\[
-g + \lambda S \, \delta\btheta = 0 \quad \Rightarrow \quad S \, \delta\btheta = -\frac{1}{\lambda} g.
\]
将 $1/\lambda$ 吸收到学习率 $\eta$ 中,即得 SR 方程 (20)。

\subsection{与经典自然梯度的严格类比}
比较经典自然梯度 $I(\btheta) \, \delta\btheta = -\eta g$ 与量子 SR $S(\btheta) \, \delta\btheta = -\eta g$,可见两者具有完全相同的形式,只是度量张量不同:
\begin{itemize}
    \item 经典:$M = I(\btheta)$(Fisher 信息矩阵)$\leftarrow$ KL 散度二阶展开 $\leftarrow$ 概率分布流形;
    \item 量子:$M = S(\btheta)$(QGT 实部)$\leftarrow$ FS 距离二阶展开 $\leftarrow$ 射影 Hilbert 空间。
\end{itemize}

\subsection{数值实现要点:奇异性、正则化与对称化}
由于参数冗余或规范自由度,$S$ 往往病态或半正定而不可逆。实践中常见处理包括:
\begin{itemize}
    \item Tikhonov 正则化:$S \to S + \lambda I$;
    \item 截断/伪逆:对小特征值截断;
    \item 对称化:用 $\frac{1}{2}(S + S^{\top})$ 或取实部 $\Re S$ 以保证对称半正定。
\end{itemize}
这些不改变几何主干:SR 的本质是"用内禀度量修正梯度方向"。

\begin{table}[ht]\centering
\caption{经典与量子信息几何的关键对应关系\label{tab:compare1}}
\begin{tabular}{p{3.0cm} p{5.8cm} p{5.8cm}}
\toprule[1pt]
\textbf{概念} & \textbf{经典概率模型} & \textbf{量子纯态模型} \\
\midrule[0.5pt]
"状态"对象 & 概率分布 $p_{\theta}(x)$ & 态向量/射线 $|\Psi(\theta)\rangle \sim c|\Psi(\theta)\rangle$ \\
输出的可观测量 & 概率本身可观测(抽样) & 振幅不可直接观测;概率由 $P(x) = |\Psi(x)|^2 / \langle\Psi|\Psi\rangle$ 给出 \\
冗余/规范 & 主要是重参数化(坐标变换) & 既有重参数化,也有全局相位/整体尺度等规范冗余 \\
"距离/散度" & KL 散度 $D_{\mathrm{KL}}(p_{\theta} \| p_{\theta + \delta})$ & FS 距离 $\mathcal{D}_{\mathrm{FS}}(\psi, \psi') = \arccos |\langle\psi|\psi'\rangle|$ \\
局部二阶展开 & $D_{\mathrm{KL}} \approx \frac{1}{2} \delta \theta^{\top} I \delta \theta$ & $\mathrm{d}s_{\mathrm{FS}}^2 = \delta \theta^{\top} g \delta \theta$(差常数因子) \\
黎曼度规 & Fisher 信息矩阵 $I_{ij} = \mathbb{E}[s_i s_j]$ & QGT $Q_{\mu\nu} = \langle \partial_\mu \psi | (I - |\psi\rangle \langle \psi|) \partial_\nu \psi \rangle$ \\
"协方差"表述 & score 协方差(对 $p$ 取期望) & log-derivative 协方差(对 $|\Psi|^2$ 取期望)+规范投影 \\
额外的反对称结构 & 无(常见 FIM 为实对称) & Berry 曲率 $F_{\mu\nu} = 2 \Im Q_{\mu\nu}$ \\
最速下降更新 & 固定 Fisher 范数 $\Rightarrow$ 自然梯度 $I \delta \theta = -\eta \nabla J$ & 固定 FS 范数 $\Rightarrow$ SR/QNG $S \delta \theta = -\eta \nabla E$ \\
\bottomrule[1pt]
\end{tabular}
\end{table}

\subsection{表 2:度量、优化方法与流形视角(含 Adam/RMSProp 的对角近似)}

\begin{table}[ht]\centering
\caption{优化算法、度量/预条件与流形解释的统一视角\label{tab:compare2}}
\begin{tabular}{p{2.8cm} p{5.5cm} p{6.2cm}}
\toprule[1pt]
\textbf{方法} & \textbf{隐含度量/预条件 $M(\theta)$} & \textbf{流形/解释} \\
\midrule[0.5pt]
SGD / GD & $M = I$(欧氏) & 把参数空间当作欧氏空间;步长与坐标有关,不具备重参数化不变性。 \\
动量法 & $M \approx I +$ 动量累积 & 仍是欧氏几何,但用一阶滤波改善病态方向的振荡。 \\
自然梯度 (NGD) & $M = I(\theta)$ (FIM) & 统计流形 (Fisher-Rao 度量);等价于固定 KL 二阶近似的最速下降。 \\
K-FAC / GGN 类方法 & $M \approx$ (FIM 或 GGN 的结构化近似) & 近似捕捉参数耦合的曲率/信息;在深度网络中常用块对角/克罗内克分解近似。 \\
RMSProp / AdaGrad & $M = \mathrm{diag}(v)$(梯度二阶矩的对角) & 经验上可看作对角预条件:当损失为 NLL 且在模型分布上取期望时,$\mathbb{E}[g_i^2]$ 与 FIM 对角元同阶,因此近似"对角 Fisher"。 \\
Adam & $M = \mathrm{diag}(\hat{v})$(带偏差校正的一阶/二阶矩对角) & 可视为 RMSProp + 动量;依然是对角近似(忽略参数相关性),但数值鲁棒且易用。 \\
牛顿法 & $M = \nabla^2 J(\theta)$ (Hessian) & 在欧氏参数空间用二阶曲率;非凸时 Hessian 可不定,需阻尼/近似。 \\
SR / QNG & $M = S \approx g_{\mathrm{FS}}$(QGT 实部/对称化) & 射影 Hilbert 空间 (FS 度量);等价于固定 FS 范数的最速下降 (TDVP 离散化)。 \\
镜像下降 & $M$ 由 Bregman 散度诱导 & 一般化"先选散度再做最速下降"的框架;KL 与 FS 都是其中的重要实例。 \\
\bottomrule[1pt]
\end{tabular}
\end{table}

\section*{结语}
本文把经典的 Fisher-Rao 信息几何与量子态的 FS/QGT 几何放在同一框架下:先选对"距离/散度",再用二阶展开得到度量,并用该度量定义最速下降。经典的 $\mathrm{KL} \Rightarrow \mathrm{Fisher} \Rightarrow$ 自然梯度;量子的 $\mathrm{FS} \Rightarrow \mathrm{QGT} \Rightarrow \mathrm{SR}/\mathrm{QNG}$。从算法角度看,许多优化方法都可理解为在更新方程 $M(\theta) \delta \theta = -\eta g$ 中选择不同的 $M$:从欧氏单位阵到 Fisher/QGT,再到各种对角/结构化近似。

\begin{thebibliography}{99}
\bibitem{Amari1998} S.-i. Amari, \textit{Natural Gradient Works Efficiently in Learning}, Neural Computation \textbf{10}(2), 251–276 (1998).
\bibitem{Sorella1998} S. Sorella, \textit{Green Function Monte Carlo with Stochastic Reconfiguration}, Phys. Rev. Lett. \textbf{80}, 4558 (1998).
\bibitem{Sorella2005} S. Sorella, \textit{Wave function optimization in the variational Monte Carlo method}, Phys. Rev. B \textbf{71}, 241103(R) (2005).
\bibitem{Provost1980} J. Provost and G. Vallée, \textit{Riemannian structure on manifolds of quantum states}, Commun. Math. Phys. \textbf{76}, 289–301 (1980).
\bibitem{Izaac2019} J. Izaac, C. Wang, and Z. Wang, \textit{Quantum Natural Gradient}, arXiv:1811.08451 (2019).
\end{thebibliography}

\end{document}
